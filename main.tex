\documentclass[11pt]{article}
\usepackage[utf8]{inputenc}
\usepackage{amsmath, amssymb}

\begin{document}

\title{David Guichard's Multivariable Calculus, late transcendentals}
\author{A Milman}
\date{\today}
\maketitle

\tableofcontents
\newpage

\section{Analytic Geometry}

In Euclidean \(n\)-space, the generalization of the Pythagorean theorem 
(and hence of circles, spheres, and hyperspheres) always involves adding more 
\emph{squared coordinate differences}, but the \emph{root itself} 
remains a \textbf{square root}, not a cube root or higher.

\subsection{The pattern in all dimensions}

For a center point
\[
\hat{\mathbf{x}} = (\hat{x}_1, \hat{x}_2, \dots, \hat{x}_n)
\]
and any point
\[
\mathbf{x} = (x_1, x_2, \dots, x_n),
\]
the Euclidean distance between them is
\[
d(\mathbf{x}, \hat{\mathbf{x}}) = 
\sqrt{(x_1 - \hat{x}_1)^2 + (x_2 - \hat{x}_2)^2 + \dots + (x_n - \hat{x}_n)^2}.
\]

The sphere (or hypersphere) of radius \(r\) centered at \(\hat{\mathbf{x}}\) is 
the set of all points where
\[
d(\mathbf{x}, \hat{\mathbf{x}}) = r,
\]
that is,
\[
(x_1 - \hat{x}_1)^2 + (x_2 - \hat{x}_2)^2 + \dots + (x_n - \hat{x}_n)^2 = r^2.
\]

\subsection{Why the root stays square}

The root remains square because Euclidean geometry measures length 
using the \emph{2-norm} (or \(L^2\) norm):
\[
\|\mathbf{x}\|_2 = \sqrt{x_1^2 + x_2^2 + \dots + x_n^2}.
\]
The ``2'' in \(L^2\) comes from squaring the coordinates; the square root 
undoes that squaring.

If one were to use a cube root instead, that would define the 
\(L^3\) norm:
\[
\|\mathbf{x}\|_3 = \sqrt[3]{|x_1|^3 + |x_2|^3 + \dots + |x_n|^3},
\]
but that is not the Euclidean distance. The corresponding ``unit ball'' would 
be a rounded box, not a sphere.

\subsection{Dimensional examples}

\begin{center}
\begin{tabular}{c|c|c|c}
\textbf{Dimension \(n\)} & \textbf{Object} & \textbf{Equation} & \textbf{Shape} \\
\hline
1 & Interval & \( |x - \hat{x}| = r \) & Line segment \\
2 & Circle & \( (x - \hat{x})^2 + (y - \hat{y})^2 = r^2 \) & Circle in a plane \\
3 & Sphere & \( (x - \hat{x})^2 + (y - \hat{y})^2 + (z - \hat{z})^2 = r^2 \) & 3D sphere \\
\(n\) & Hypersphere & \( \sum_{i=1}^n (x_i - \hat{x}_i)^2 = r^2 \) & \(n\)-dimensional surface
\end{tabular}
\end{center}

\subsection{Summary insight}

\begin{itemize}
  \item The sum of squares reflects orthogonality in Euclidean geometry.
  \item The square root comes from the Pythagorean theorem.
  \item Increasing the number of dimensions adds more squared terms,
        not higher powers or roots.
\end{itemize}

Thus, whether describing a \textbf{circle (2D)}, \textbf{sphere (3D)},
or \textbf{hypersphere (\(n\)D)}, the defining equation always involves
\emph{squares and a single square root}, never cubic or higher roots.



\end{document}

